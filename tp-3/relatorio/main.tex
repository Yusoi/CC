% ---------------------------------------------------------------------------- %

\documentclass{llncs}
\usepackage[utf8]{inputenc}
\usepackage[T1]{fontenc}
\usepackage[portuges]{babel}
\usepackage[hidelinks]{hyperref}
\usepackage{times}
\usepackage{a4}
\usepackage{epstopdf}
\usepackage{graphicx}
\usepackage{fancyvrb}
\usepackage{amsmath}
\usepackage{amssymb}
\usepackage[dvipsnames]{xcolor}
\usepackage{xspace}
\usepackage{xpatch}
\usepackage{minted}

% ---------------------------------------------------------------------------- %
% utilities

\newcommand{\itemizedpar}[1]{\paragraph{\emph{\textbf{#1}}}}

\newcommand{\questao}[2]{%
  \itemizedpar{Questão #1)}%
  \emph{#2}%
  \vspace*{0.6\baselineskip}%
  }

\newcommand{\SYS}{\textsc{FileShare}\xspace}

\makeatletter
\AtBeginEnvironment{minted}{\dontdofcolorbox}
\def\dontdofcolorbox{\renewcommand\fcolorbox[4][]{##4}}
\xpatchcmd{\inputminted}{\minted@fvset}{\minted@fvset\dontdofcolorbox}{}{}
\makeatother

\setminted{
  autogobble,
  breaklines,
  fontsize=\small,
  frame=leftline,
  framesep=8pt,
  xleftmargin=11pt
  }

\let\oldurl\url
\renewcommand{\url}[1]{{\color{Blue}\oldurl{#1}}}

% ---------------------------------------------------------------------------- %
% comments

\newcommand{\mynote}[3]{
   \fbox{\bfseries\sffamily\scriptsize#1}
   {\scriptsize$\blacktriangleright$\textsf{\emph{\color{#2}{#3}}}$\blacktriangleleft$}}

\newcommand{\alberto}[1]{\mynote{alberto}{red}{#1}}
\newcommand{\bernardo}[1]{\mynote{cesar}{red}{#1}}
\newcommand{\nuno}[1]{\mynote{luis}{red}{#1}}

% ---------------------------------------------------------------------------- %
% document

\begin{document}

\mainmatter

\title{TP3: Serviço de Resolução de Nomes (DNS)}

\author{
  {\bfseries Grupo PL59} \\
  Alberto Campinho Faria \\
  Bernardo Manuel Ribeiro Marques Soares Silva \\
  Nuno Filipe Maranhão dos Reis
  }

\institute{
  University of Minho, Department of  Informatics, 4710-057 Braga, Portugal \\
  e-mail: \{a79077,a77230,a77310\}@alunos.uminho.pt
  }

\date{}

\maketitle

% ---------------------------------------------------------------------------- %

\section{Introdução}

Este documento consiste no relatório correspondente ao segundo trabalho prático desenvolvido no âmbito da Unidade Curricular de Comunicações por Computador do curso de Mestrado Integrado em Engenharia Informática, no ano letivo de 2018/2019, da Universidade do Minho.

Com o trabalho em questão, pretende-se que seja construído um sistema \nonpt{peer-to-peer} transferência fiável de ficheiros, cada \nonpt{peer} podendo iniciar transferências de \nonpt{download} e \nonpt{upload}. O protocolo base de transferência de dados deve garantir a entrega sem erros e ordenada de mensagens, no entanto devendo também utilizar somente conexões UDP.

Neste sentido, foi desenvolvido o sistema \SYS, o qual cumpre todos os requisitos impostos pelo enunciado do trabalho. O sistema disponibiliza ainda várias funcionalidades adicionais, destacando-se \inlineenum{1} a possibilidade de realizar transferências em simultâneo entre qualquer par de \nonpt{peers} (multiplexando várias conecções no mesmo socket UDP) e \inlineenum{2} a paralelização do \nonpt{download} de ficheiros utilizando vários \nonpt{peers}.

\itemizedpar{Material disponibilizado.}

A implementação do sistema, na linguagem Java e sob a forma de um projeto do IDE \emph{IntelliJ IDEA}\footnote{\url{https://www.jetbrains.com/idea/}}, é incluída no arquivo \ARQUIVO, disponibilizado juntamente com este relatório. O arquivo contém ainda o programa \path{fileshare.jar}, resultado da compilação do projeto, através do qual se pode utilizar o sistema. Para executar este programa, recomenda-se a utilização do \nonpt{script} \path{fileshare.sh}, o qual adiciona funcionalidades de histórico de comandos ao \nonpt{prompt} interativo do programa.

\itemizedpar{Estrutura do documento.}

Na Secção~\ref{sec:func} são descritas todas as funcionalidades disponibilizadas pelo sistema e o modo de utilização da interface de linha de comandos do mesmo. Na Secção~\ref{sec:impl} é depois apresentada a arquitetura interna do sistema e detalhados vários aspetos relativos à sua implementação, incluindo o protocolo de comunicação fiável sobre UDP. A Secção~\ref{sec:conc} conclui o relatório.

% Na Secção~\ref{sec:teste} é descrito o processo de teste do sistema e apresentados os respetivos resultados. 

% ---------------------------------------------------------------------------- %

% ---------------------------------------------------------------------------- %

\section{Questões e Respostas}
\label{sec:respostas}

% ---------------------------------------------------------------------------- %

\questao{a}{Qual o conteúdo do ficheiro \emph{/etc/resolv.conf} e para que serve essa informação?}

\noindent O ficheiro \emph{/etc/resolv.conf} possui o seguinte conteúdo:

\begin{minted}{text}
  # Dynamic resolv.conf(5) file for glibc resolver(3) generated by resolvconf(8)
  #     DO NOT EDIT THIS FILE BY HAND -- YOUR CHANGES WILL BE OVERWRITTEN
  nameserver 10.0.2.3
  search eduroam.uminho.pt
\end{minted}

Este ficheiro contém informação de configuração para o serviço de resolução de nomes de domínio empregado pelo sistema operativo. O seu conteúdo tem o seguinte significado:

\begin{itemize}

    \item A linha \path{nameserver 10.0.2.3} identifica o endereço IP do servidor de nomes que deve ser usado para servir pedidos de resolução de nomes;

    \item A linha \path{search eduroam.uminho.pt} especifica o modo de resolução de nomes que não contenham nenhum ponto (tendo em conta que a opção \path{ndots} tem o valor por defeito de 1). Por exemplo, para o nome \path{abc} e tendo em conta esta linha, deverá ser feita a tentativa de resolver o nome como \path{abc.eduroam.uminho.pt}.

\end{itemize}

% ---------------------------------------------------------------------------- %

\questao{b}{Os servidores \path{www.google.pt.} e \path{www.google.com.} têm endereços IPv6? Se sim, quais?}

\noindent Para determinar os endereços dos servidores em questão, foi utilizado o comando \path{host} da seguinte forma:

\begin{minted}{console}
    $ host www.google.pt.
    www.google.pt has address 216.58.214.163
    www.google.pt has IPv6 address 2a00:1450:4003:801::2003
    $ host www.google.com.
    www.google.com has address 172.217.16.228
    www.google.com has IPv6 address 2a00:1450:4003:803::2004
\end{minted}

Pelo output destes comandos determina-se que ambos os servidores têm, de facto, endereços IPv6:

\begin{itemize}
    \item O servidor \path{www.google.pt.} tem endereço IPv6 \path{2a00:1450:4009:809::2003};
    \item O servidor \path{www.google.com.} tem endereço IPv6 \path{2a00:1450:4009:809::2004}.
\end{itemize}

% ---------------------------------------------------------------------------- %

\questao{c}{Quais os servidores de nomes definidos para os domínios: ``\path{ccg.pt.}'', ``\path{pt.}'' e ``\path{.}''?}

\noindent Os servidores de nomes definidos para o domínio ``\path{ccg.pt.}'' são os seguintes:

\begin{itemize}
    \item \path{ns1.ccg.pt.} com endereço IPv4 \path{193.136.11.201};
    \item \path{ns3.ccg.pt.} com endereço IPv4 \path{193.136.11.203}.
\end{itemize}

\noindent Os servidores de nomes definidos para o domínio ``\path{pt.}'' são os seguintes:

\begin{itemize}
    \item \path{a.dns.pt.} com endereço IPv4 \path{185.39.208.1};
    \item \path{b.dns.pt.} com endereço IPv4 \path{194.0.25.23};
    \item \path{c.dns.pt.} com endereço IPv4 \path{204.61.216.105};
    \item \path{d.dns.pt.} com endereço IPv4 \path{185.39.210.1};
    \item \path{e.dns.pt.} com endereço IPv4 \path{193.136.192.64};
    \item \path{f.dns.pt.} com endereço IPv4 \path{162.88.45.1};
    \item \path{g.dns.pt.} com endereço IPv4 \path{193.136.2.226};
    \item \path{ns.dns.br.} com endereço IPv4 \path{200.160.0.5};
    \item \path{ns2.nic.fr.} com endereço IPv4 \path{192.93.0.4};
    \item \path{sns-pb.isc.org.} com endereço IPv4 \path{192.5.4.1}.
\end{itemize}

\noindent Os servidores de nomes definidos para o domínio ``\path{.}'' são os seguintes:

\begin{itemize}
    \item \path{a.root-servers.net.} com endereço IPv4 \path{198.41.0.4};
    \item \path{b.root-servers.net.} com endereço IPv4 \path{199.9.14.201};
    \item \path{c.root-servers.net.} com endereço IPv4 \path{192.33.4.12};
    \item \path{d.root-servers.net.} com endereço IPv4 \path{199.7.91.13};
	\item \path{e.root-servers.net.} com endereço IPv4 \path{192.203.230.10};
	\item \path{f.root-servers.net.} com endereço IPv4 \path{192.5.5.241};
	\item \path{g.root-servers.net.} com endereço IPv4 \path{192.112.36.4};
	\item \path{h.root-servers.net.} com endereço IPv4 \path{198.97.190.53};
	\item \path{i.root-servers.net.} com endereço IPv4 \path{192.36.148.17};
	\item \path{j.root-servers.net.} com endereço IPv4 \path{192.58.128.30};
	\item \path{k.root-servers.net.} com endereço IPv4 \path{193.0.14.129};
	\item \path{l.root-servers.net.} com endereço IPv4 \path{199.7.83.42};
	\item \path{m.root-servers.net.} com endereço IPv4 \path{202.12.27.33}.
\end{itemize}

% ---------------------------------------------------------------------------- %

\questao{d}{Existe o domínio \path{eureka.software.}? Será que \path{eureka.software.} é um host?}

\noindent O domínio \path{eureka.software.} existe e é um host, como se pode determinar através do output do seguinte comando:

\begin{minted}{console}
    $ nslookup
    > set type=a
    > eureka.software.
    Server:  127.0.1.1
    Address: 127.0.1.1#53

    Non-authoritative answer:
    Name:    eureka.software
    Address: 34.214.90.141
\end{minted}

% ---------------------------------------------------------------------------- %

\questao{e}{Qual é o servidor DNS primário definido para o domínio \path{ami.pt.}? Este servidor primário (\emph{master}) aceita queries recursivas? Porquê?}

\noindent O servidor DNS primário definido para o domínio \path{ami.pt.} tem endereço IPv4 \path{80.172.230.28} (\path{ns1.dot2web.com.}):

\begin{minted}{console}
    $ host -t NS ami.pt
    ami.pt name server ns1.dot2web.com.
    ami.pt name server ns2.dot2web.com.
\end{minted}

Este servidor não aceita queries recursivas, uma vez que a flag \path{RA} não está presente no seguinte header de resposta obtido através do comando \path{dig @ns1.dot2web.com google.com}:

\begin{minted}{text}
    ;; ->>HEADER<<- opcode: QUERY, status: REFUSED, id: 51127
    ;; flags: qr rd; QUERY: 1, ANSWER: 0, AUTHORITY: 0, ADDITIONAL: 1
    ;; WARNING: recursion requested but not available
\end{minted}

% ---------------------------------------------------------------------------- %

\questao{f}{Obtenha uma resposta ``\emph{autoritativa}'' para a questão anterior.}

(A resposta na questão anterior é autoritativa.)

% ---------------------------------------------------------------------------- %

\questao{g}{Onde são entregues as mensagens dirigidas a \path{marcelo@presidencia.pt}? E a \path{guterres@onu.org}?}

\noindent As mensagens dirigidas a \path{marcelo@presidencia.pt} são entregues em \path{mail1.presidencia.pt.} ou \path{mail2.presidencia.pt.}:

\begin{minted}{console}
    $ dig -t MX presidencia.pt
    ; <<>> DiG 9.11.4-3ubuntu5.1-Ubuntu <<>> -t MX presidencia.pt
    ;; global options: +cmd
    ;; Got answer:
    ;; ->>HEADER<<- opcode: QUERY, status: NOERROR, id: 25743
    ;; flags: qr rd ra; QUERY: 1, ANSWER: 2, AUTHORITY: 0, ADDITIONAL: 1
    
    ;; OPT PSEUDOSECTION:
    ; EDNS: version: 0, flags:; udp: 65494
    ;; QUESTION SECTION:
    ;presidencia.pt.			IN	MX
    
    ;; ANSWER SECTION:
    presidencia.pt.		86400	IN	MX	50 mail1.presidencia.pt.
    presidencia.pt.		86400	IN	MX	10 mail2.presidencia.pt.
    
    ;; Query time: 88 msec
    ;; SERVER: 127.0.0.53#53(127.0.0.53)
    ;; WHEN: sex abr 12 22:45:58 WEST 2019
    ;; MSG SIZE  rcvd: 87
\end{minted}

\noindent As mensagens dirigidas a \path{guterres@onu.org} são entregues em \path{mail.onu.org.}:

\begin{minted}{console}
    $ dig -t MX onu.org
    ; <<>> DiG 9.11.4-3ubuntu5.1-Ubuntu <<>> -t MX onu.org
    ;; global options: +cmd
    ;; Got answer:
    ;; ->>HEADER<<- opcode: QUERY, status: NOERROR, id: 64402
    ;; flags: qr rd ra; QUERY: 1, ANSWER: 1, AUTHORITY: 0, ADDITIONAL: 1
    
    ;; OPT PSEUDOSECTION:
    ; EDNS: version: 0, flags:; udp: 65494
    ;; QUESTION SECTION:
    ;onu.org.			IN	MX
    
    ;; ANSWER SECTION:
    onu.org.		3600	IN	MX	10 mail.onu.org.
    
    ;; Query time: 825 msec
    ;; SERVER: 127.0.0.53#53(127.0.0.53)
    ;; WHEN: sex abr 12 22:47:15 WEST 2019
    ;; MSG SIZE  rcvd: 57
\end{minted}

% ---------------------------------------------------------------------------- %

\questao{h}{Que informação é possível obter acerca de \path{www.whitehouse.gov}? Qual é o endereço IPv4 associado?}

\noindent O endereço IPv4 associado (após resolver \emph{aliases}) é modificado regularmente. No momento da escrita, o endereço IPv4 era \path{2.19.159.149}.

Realizando-se queries dos tipos A, AAAA, CNAME, MX, NS, PTR, SIG, SOA, SRV e TXT, apenas foi possível reunir a seguinte informação:

\begin{minted}{text}
    www.whitehouse.gov.	23	IN	CNAME	wildcard.whitehouse.gov.edgekey.net.
    wildcard.whitehouse.gov.edgekey.net. 313 IN CNAME e4036.dscb.akamaiedge.net.
    e4036.dscb.akamaiedge.net. 19	IN	A	23.64.18.121
    e4036.dscb.akamaiedge.net. 20	IN	AAAA	2a02:26f0:a1:297::fc4
    e4036.dscb.akamaiedge.net. 20	IN	AAAA	2a02:26f0:a1:2ba::fc4
\end{minted}

% ---------------------------------------------------------------------------- %

\questao{i}{Consegue interrogar o DNS sobre o endereço IPv6 \path{2001:690:a00:1036:1113::247} usando algum dos clientes DNS? Que informação consegue obter? Supondo que teve problemas com esse endereço, consegue obter um contacto do responsável por esse IPv6?}

\noindent Utilizando o comando \texttt{host}, determinou-se que o endereço IPv6 em questão corresponde ao endereço \path{www.fccn.pt.}. Um contato do responsável por esse endereço é \path{dns-tec@fccn.pt}.

% ---------------------------------------------------------------------------- %

\questao{j}{Os secundários usam um mecanismo designado por ``Transferência de zona'' para se atualizarem automaticamente a partir do primário, usando os parâmetros definidos no Record do tipo SOA do domínio. Descreve sucintamente esse mecanismo com base num exemplo concreto (ex: di.uminho.pt ou o domínio cc.pt que vai ser criado na topologia virtual).}

\noindent O mecanismo de transferência de zona permite replicar \emph{zonas DNS} --- porções contíguas do espaço de nomes --- entre servidores DNS. As transferência de zona são iniciadas pelos secundários; no entanto, o primário pode notificar os secundários de que foram realizadas alterações à informação da zona através de mensagens NOTIFY.

Existem dois tipos de transferência de zona: \emph{completa} e \emph{incremental}. Com transferências de zona completas, toda a informação da zona é transmitida ao servidor que iniciou a transferência. Com transferências de zona incrementais, no entanto, apenas a informação desatualizada é transmitida.

Para iniciar uma transferência de zona, o secundário começa por obter o record \emph{Start of Authority} (SOA) do primário, verificando depois se o \emph{serial number} contido nesse record é superior àquele da cópia da zona que o secundário detém. Se tal não se verificar (\emph{i.e.}, a informação no secundário está atualizada), o processo é interrompido.

Se, de facto, o \emph{serial number} obtido for superior ao atual do secundário, conclui-se que a cópia da zona detida pelo secundário está desatualizada e é iniciada então a fase de transmissão do conteúdo da zona do primário para o secundário. Para tal, no caso de se desejar efetuar uma transferência de zona completa, o secundário envia uma query de tipo AXFR ao primário, o qual responde com várias mensagens compondo os records de todos os domínios na zona. Caso se pretenda efetuar uma transferência de zona incremental, o secundário envia uma query de tipo IXFR, na qual especifica o \emph{serial number} da sua cópia da zona, após a qual o secundário responde apenas com os records dos domínios alterados desde essa versão.

Os seguintes passos representam um possível exemplo de uma transferência de zona entre o servidor de nomes primário \path{dns.di.uminho.pt} e o secundário \path{dns2.di.uminho.pt}:

\begin{enumerate}
    \item Como consequência, \path{dns2.di.uminho.pt} pede o record SOA de \path{dns.di.uminho.pt};
    \item Comparando o \emph{serial number} no record SOA com o da cópia local da zona, determina-se que a cópia está desatualizada;
    \item \path{dns2.di.uminho.pt} envia uma query AFXR a \path{dns.di.uminho.pt};
    \item \path{dns.di.uminho.pt} responde com várias mensagens compondo os records de todos os domínios na zona.
\end{enumerate}

% ---------------------------------------------------------------------------- %

% % ---------------------------------------------------------------------------- %

\section{Criação de um domínio de nomes \texttt{CC.PT}}
\label{sec:dominio}

Pretende-se que crie um domínio CC.PT para a topologia de rede que estamos a usar nas aulas práticas (CC-Topo-
2019.imn), de modo a que se possam usar os nomes em vez dos endereços IP. No final deve, por exemplo, poder fazer-se “ ping
servidor1.cc.pt ” ou mesmo apenas “ ping Servidor1 ” ou “ ping Servidor1.cc.pt. ”em vez de “ ping 10.1.1.1 ”.

Requisitos a cumprir:

\begin{itemize}

  \item Criação do domínio cc.pt co servidor primário em Servidor1 10.1.1.1 e secundário em Urano 10.2.2.3
  
  \item Criação do domínio reverso 1.1.10.in-addr.arpa com os mesmos servidores
  
  \item O servidor primário do domínio é o “Servidor1” com endereço 10.1.1.1, também designado por dns.cc.pt, tendo como secundário o “Urano” com endereço 10.2.2.3, com alias dns2.cc.pt. O administrador do domínio é o grupoXX@cc.pt (onde XX é o número do grupo).

  \item O domínio tem também um servidor Web (www.cc.pt) e um servidor de e-mail principal (mail.cc.pt) em Servidor3. O servidor pop e imap é o Servidor2, que é também servidor secundário do e-mail para o domínio;

  \item Sem prejuízo de outros registos que se possam considerar, devem estar registados também o Cliente1.cc.pt com alias GrupoXX.cc.pt onde XX é o número do grupo, e Alfa.cc.pt, Beta.cc.pt e Gama.cc.pt no domínio de nomes e no domínio reverso.
  
\end{itemize}

% ---------------------------------------------------------------------------- %

% ---------------------------------------------------------------------------- %

\section{Conclusão}
\label{sec:conclusao}

Este relatório apresentou uma resolução ao terceiro trabalho prático desenvolvido no âmbito da Unidade Curricular de Comunicações por Computador do curso de Mestrado Integrado em Engenharia Informática, no ano letivo de 2018/2019, da Universidade do Minho.

Com este trabalho estudaram-se o modo de utilização e os mecanismos de funcionamento do serviço DNS de resolução de nomes. Todas as questões propostas pelo enunciado do trabalho foram respondidas. A criação e configuração do domínio de nomes \path{CC.PT} foi também realizada e todos os requisitos identificados pelo enunciado foram cumpridos.

% ---------------------------------------------------------------------------- %


\end{document}

% ---------------------------------------------------------------------------- %
