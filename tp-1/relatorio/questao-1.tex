% ---------------------------------------------------------------------------- %

\questao{1}{Inclua no relatório uma tabela em que identifique, para cada comando
executado, qual o protocolo de aplicação, o protocolo de transporte, porta de
atendimento e overhead de transporte.}

\newcommand{\multilinecell}[1]{
  {\renewcommand{\arraystretch}{1.0}
  \rule{0pt}{17pt}
  \begin{tabular}{@{}c@{}}#1\end{tabular}
  \rule[-12pt]{0pt}{0pt}}
  }

\begin{table}[ht]
  \centering
  \setlength{\tabcolsep}{7pt}
  \renewcommand{\arraystretch}{1.3}
  \begin{tabular}{ | c | c | c | c | c | }

    \hline

    \textbf{Comando usado} & \multilinecell{\textbf{Protocolo de} \\ \textbf{aplicação}} & \multilinecell{\textbf{Protocolo de} \\ \textbf{transporte}} & \multilinecell{\textbf{Porta de} \\ \textbf{atendimento}} & \multilinecell{\textbf{\emph{Overhead} de} \\ \textbf{transporte}} \\ \hline

    \texttt{Ping} & --- & --- & --- & --- \\ \hline

    \texttt{traceroute} & --- & --- & --- & --- \\ \hline

    \texttt{telnet} & --- & --- & --- & --- \\ \hline

    \texttt{ftp} & --- & --- & --- & --- \\ \hline

    \texttt{Tftp} & --- & --- & --- & --- \\ \hline

    \texttt{browser/http} & --- & --- & --- & --- \\ \hline

    \texttt{nslookup} & --- & --- & --- & --- \\ \hline

    \texttt{ssh} & --- & --- & --- & --- \\ \hline

  \end{tabular}
\end{table}

% ---------------------------------------------------------------------------- %
