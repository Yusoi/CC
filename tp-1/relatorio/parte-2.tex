% ---------------------------------------------------------------------------- %

\section*{Parte 2 - Secção 2}

% \begin{figure}[ht]
%   \centering
%   \includegraphics[width=\textwidth]{figures/parte2-seccao2-topologia.png}
%   \caption{Topologia CORE para a secção 2 da parte 2.}
%   \label{fig:parte2-seccao2-topologia}
% \end{figure}

A topologia CORE construída no âmbito deste exercício, de acordo com as
instruções dadas no enunciado do trabalho, é apresentada na
Figura~\ref{fig:parte2-seccao2-topologia}.

\itemizedpar{Alínea 1.a.}

Os endereços IP e máscaras de rede atribuídos pelo CORE a cada equipamento são
visíveis na Figura~\ref{fig:parte2-seccao2-topologia}.

\itemizedpar{Alínea 1.b.}

Os endereços em questão são privados, uma vez que, pela norma RFC
1918\footnote{\url{https://tools.ietf.org/html/rfc1918}}, endereços na gama
10.0.0.0 - 10.255.255.255 (i.e., 10.0.0.0/8) são privados.

\itemizedpar{Alínea 1.c.}

Não são atribuídos endereços IP aos \emph{switches} uma vez que estes operam
inteiramente abaixo da camada de rede.

\itemizedpar{Alínea 1.d.}

Existe, de facto, conectividade IP entre os \emph{laptops} dos vários
departamentos e o servidor do departamento C (o qual tem endereço IP 10.0.4.10).

Output do comando \emph{ping} no \emph{laptop} \emph{La1}:

\begin{Verbatim}[fontsize=\scriptsize]
    PING 10.0.4.10 (10.0.4.10) 56(84) bytes of data.
    64 bytes from 10.0.4.10: icmp_req=1 ttl=62 time=3.18 ms
    64 bytes from 10.0.4.10: icmp_req=2 ttl=62 time=0.963 ms
    64 bytes from 10.0.4.10: icmp_req=3 ttl=62 time=0.981 ms
\end{Verbatim}

Output do comando \emph{ping} no \emph{laptop} \emph{Lb1}:

\begin{Verbatim}[fontsize=\scriptsize]
    PING 10.0.4.10 (10.0.4.10) 56(84) bytes of data.
    64 bytes from 10.0.4.10: icmp_req=1 ttl=62 time=3.04 ms
    64 bytes from 10.0.4.10: icmp_req=2 ttl=62 time=0.954 ms
    64 bytes from 10.0.4.10: icmp_req=3 ttl=62 time=0.993 ms
\end{Verbatim}

Output do comando \emph{ping} no \emph{laptop} \emph{Lc1}:

\begin{Verbatim}[fontsize=\scriptsize]
    PING 10.0.4.10 (10.0.4.10) 56(84) bytes of data.
    64 bytes from 10.0.4.10: icmp_req=1 ttl=64 time=1.21 ms
    64 bytes from 10.0.4.10: icmp_req=2 ttl=64 time=0.355 ms
    64 bytes from 10.0.4.10: icmp_req=3 ttl=64 time=0.398 ms
\end{Verbatim}

\itemizedpar{Alínea 1.e.}

Existe, de facto, conectividade IP entre o router \emph{Rext} e o servidor
\emph{S1} (o qual tem endereço IP 10.0.4.10).

Output do comando \emph{ping} no router \emph{Rext}:

\begin{Verbatim}[fontsize=\scriptsize]
    PING 10.0.4.10 (10.0.4.10) 56(84) bytes of data.
    64 bytes from 10.0.4.10: icmp_req=1 ttl=63 time=2.16 ms
    64 bytes from 10.0.4.10: icmp_req=2 ttl=63 time=0.690 ms
    64 bytes from 10.0.4.10: icmp_req=3 ttl=63 time=0.691 ms
\end{Verbatim}

\itemizedpar{Alínea 2.a.}

Resultado do comando \texttt{netstat -rn} no router \emph{Ra}:

\begin{Verbatim}[fontsize=\scriptsize]
    Kernel IP routing table
    Destination     Gateway         Genmask         Flags   MSS Window  irtt Iface
    10.0.0.0        0.0.0.0         255.255.255.0   U         0 0          0 eth0
    10.0.1.0        10.0.0.2        255.255.255.0   UG        0 0          0 eth0
    10.0.2.0        0.0.0.0         255.255.255.0   U         0 0          0 eth1
    10.0.3.0        0.0.0.0         255.255.255.0   U         0 0          0 eth2
    10.0.4.0        10.0.0.2        255.255.255.0   UG        0 0          0 eth0
    10.0.5.0        10.0.0.2        255.255.255.0   UG        0 0          0 eth0
    10.0.6.0        10.0.2.1        255.255.255.0   UG        0 0          0 eth1
\end{Verbatim}

Resultado do comando \texttt{netstat -rn} no laptop \emph{La1}:

\begin{Verbatim}[fontsize=\scriptsize]
    Kernel IP routing table
    Destination     Gateway         Genmask         Flags   MSS Window  irtt Iface
    0.0.0.0         10.0.3.1        0.0.0.0         UG        0 0          0 eth0
    10.0.3.0        0.0.0.0         255.255.255.0   U         0 0          0 eth0
\end{Verbatim}

Cada linha de ambas as tabelas pode ser interpretada da seguinte forma:

\begin{itemize}

  \item O tráfego destinado ao endereço na coluna "Destination", tendo em conta
  a coluna "Genmask", deverá ser encaminhado para o endereço na coluna
  "Gateway". Se o valor na coluna "Gateway" for 0.0.0.0, o tráfego deverá ser
  enviado diretamente para o endereço de destino.

  \item Na primeira linha da primeira tabela, por exemplo, especifica-se que o
  tráfego direcionado a endereços da forma 10.0.0.0/24 (a máscara /24 vem do
  campo "Genmask") deverá ser enviado diretamente para o endereço de destino.

  \item Como outro exemplo, a primeira linha da segunda tabela especifica que o
  tráfego direcionado a qualquer endereço que não seja da forma 10.0.3.0/24
  (especificado pela segunda linha da mesma tabela) deverá ser encaminhado para
  o gateway em 10.0.3.1. Pela segunda linha da segunda tabela, tráfego
  direcionado a qualquer endereço da forma 10.0.3.0/24 deverá ser enviado
  diretamente para o endereço de destino.

  \item O carater \texttt{U} na coluna "Flags" indica que a rota está ativa
  (\emph{up}).

  \item O carater \texttt{G} na coluna "Flags" indica que o tráfego deverá ser
  encaminhado pelo gateway especificado pela mesma linha (isto acontece quando
  esse valor não é 0.0.0.0).

  \item A coluna "Iface" especifica a interface através da qual o tráfego em
  questão deverá ser enviado.

\end{itemize}

\itemizedpar{Alínea 2.b.}

Resultado do comando \texttt{ps ax} no router \emph{Ra}:

\begin{Verbatim}[fontsize=\scriptsize]
  PID TTY      STAT   TIME COMMAND
    1 ?        S      0:00 /usr/sbin/vnoded -v -c /tmp/pycore.60756/Ra -l /tmp/pycore.60756/Ra.log -p /tmp/pycore.60756/Ra.pid -C /tmp/pycore.6
   60 ?        Ss     0:00 /usr/lib/quagga/zebra -u root -g root -d
   71 ?        Ss     0:00 /usr/lib/quagga/ospf6d -u root -g root -d
   72 ?        Ss     0:00 /usr/lib/quagga/ospfd -u root -g root -d
  192 pts/4    Ss     0:00 /bin/bash
  252 pts/4    R+     0:00 ps ax
\end{Verbatim}

Resultado do comando \texttt{ps ax} no laptop \emph{La1}:

\begin{Verbatim}[fontsize=\scriptsize]
  PID TTY      STAT   TIME COMMAND
    1 ?        S      0:00 /usr/sbin/vnoded -v -c /tmp/pycore.60756/La1 -l /tmp/
   75 pts/4    Ss     0:00 /bin/bash
  131 pts/4    R+     0:00 ps ax
\end{Verbatim}

Uma vez que o programa de roteamento \emph{quagga} está ativo no router,
conclui-se que está a ser utilizado encaminhamento dinâmico.

\itemizedpar{Alínea 2.c.}

Resultado do comando \texttt{netstat -rn} no servidor \emph{S1} antes de se
remover a rota por defeito:

\begin{Verbatim}[fontsize=\scriptsize]
    Kernel IP routing table
    Destination     Gateway         Genmask         Flags   MSS Window  irtt Iface
    0.0.0.0         10.0.4.1        0.0.0.0         UG        0 0          0 eth0
    10.0.4.0        0.0.0.0         255.255.255.0   U         0 0          0 eth0
\end{Verbatim}

Resultado do comando \texttt{netstat -rn} no servidor \emph{S1} \emph{após} se
remover a rota por defeito através do comando \texttt{route delete default}:

\begin{Verbatim}[fontsize=\scriptsize]
    Kernel IP routing table
    Destination     Gateway         Genmask         Flags   MSS Window  irtt Iface
    10.0.4.0        0.0.0.0         255.255.255.0   U         0 0          0 eth0
\end{Verbatim}

Esta medida faz com que o servidor \emph{S1} deixe de conseguir enviar mensagens
para endereços fora da rede do departamento C. Isto é devido ao facto de, após
se remover a rota por defeito, a tabela de roteamento apenas especificar como se
deve tratar tráfego de saída com destino a endereços da forma 10.0.4.0/24.

Através da utilização do comando \emph{ping} (com destino a \emph{S1}) nos
vários routers e em um laptop de cada departamento, verificou-se que a remoção
da rota por defeito tem de facto as implicações descritas.

\itemizedpar{Alínea 2.d.}

Utilizaram-se os seguintes dois comandos para adicionar as rotas necessárias
para restaurar a conetividade entre o servidor \emph{S1} e as \emph{redes dos
departamentos} (não se considerou a restauração de conetividade com o router
\emph{Rext}):

\begin{Verbatim}[fontsize=\scriptsize]
    route add -net 10.0.3.0 gw 10.0.4.1 netmask 255.255.255.0
    route add -net 10.0.6.0 gw 10.0.4.1 netmask 255.255.255.0
\end{Verbatim}

Estas regras especificam que tráfego direcionado a endereços da forma
10.0.3.0/24 (departamento A) ou 10.0.6.0/24 (departamento B) deverá ser
encaminhado para o gateway em 10.0.4.1 (router \emph{Rc}).

\itemizedpar{Alínea 2.e.}

Testou-se a conetividade ao servidor executando-se o comando \emph{ping} (com
destino a \emph{S1}) em um laptop de cada departamento e nos respetivos routers.
Verificou-se que as rotas adicionadas reestabeleceram a conetividade de todos
esses equipamentos com o servidor.

Output do comando \texttt{netstat -rn} no servidor \emph{S1} após adicionar as
rotas estáticas através dos comandos transcritos na resposta à alínea anterior
(nova tabela de encaminhamento do servidor):

\begin{Verbatim}[fontsize=\scriptsize]
    Kernel IP routing table
    Destination     Gateway         Genmask         Flags   MSS Window  irtt Iface
    10.0.3.0        10.0.4.1        255.255.255.0   UG        0 0          0 eth0
    10.0.4.0        0.0.0.0         255.255.255.0   U         0 0          0 eth0
    10.0.6.0        10.0.4.1        255.255.255.0   UG        0 0          0 eth0
\end{Verbatim}

% ---------------------------------------------------------------------------- %

\section*{Parte 2 - Secção 3}

% \begin{figure}[ht]
%   \centering
%   \includegraphics[width=\textwidth]{figures/parte2-seccao3-topologia.png}
%   \caption{Topologia CORE para a secção 3 da parte 2.}
%   \label{fig:parte2-seccao3-topologia}
% \end{figure}

A topologia CORE construída no âmbito deste exercício, de acordo com as
instruções dadas no enunciado do trabalho, é apresentada na
Figura~\ref{fig:parte2-seccao3-topologia}.

\itemizedpar{Alínea 1.}

De acordo com o definido pelo enunciado e tendo em conta que o grupo do qual os
autores deste documento fazem parte tem número 67, considerou-se que se dispunha
apenas dos endereços de rede IP 172.67.48.0/20.

Como convenção, determinou-se que os bits 20-23 dos endereço identificam o
departamento, e os bits 24-31 identificam interfaces dentro de um determinado
departamento. Esta escolha permite até 16 departamentos e simplifica a gestão de
endereços garantindo que os três primeiros componentes dos endereços IP num
determinado departamento são fixos.

Para o departamento A, atribuiu-se o endereço 172.67.48.1 à respetiva interface
do router \emph{Ra}, atribuindo-se endereços na gama 172.67.48.2---172.67.48.254
às interfaces dos hosts desse mesmo departamento (cf.
Figura~\ref{fig:parte2-seccao3-topologia}).

Para os departamentos B e C efetuaram-se procedimentos idênticos, sendo que para
o primeiro se atribuiram endereços cujos três primeiros componentes são
"172.67.49" e para o segundo "172.67.50".

\itemizedpar{Alínea 2.}

As máscaras de rede utilizadas nos três departamentos são as seguintes:

\begin{enumerate}
  \item Departamento A: 172.67.48.0/24;
  \item Departamento B: 172.67.49.0/24;
  \item Departamento C: 172.67.50.0/24.
\end{enumerate}

Em cada departamento podem-se interligar 253 hosts, uma vez que se podem alterar
os últimos 8 bits (último componente) do endereço, excluindo os endereços cujos
últimos componentes são 0, 255 e 1 (este último devido a ter sido atribuído ao
respetivo router): \(2^8 - 3 = 253\).

\itemizedpar{Alínea 3.}

Executou-se o comando ping em um laptop de cada departamento, direcionado para
um outro laptop de cada um dos restantes departamentos e também para o servidor
\emph{S1}. Verificou-se que existia conetividade entre as redes dos vários
departamentos.

% ---------------------------------------------------------------------------- %
